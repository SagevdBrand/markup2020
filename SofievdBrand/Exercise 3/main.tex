\documentclass[aspectratio=169]{beamer}
\usetheme[]{default}
\usecolortheme{beaver}
\usepackage{tikz}
\usepackage{multirow}
\usepackage{amsmath}

\beamertemplatenavigationsymbolsempty %suppress navigation bar


%TITLEPAGE
\title[] {Example document to recreate with \texttt{beamer} in \LaTeX}

\author[van den Brand]{
  Sofie van den Brand }
\date{}


\begin{document}
% ------- title page -------%
\begin{frame}
 \titlepage 
\vskip
\begin{center}
 FALL 2020 \\
Markup Languages and Reproducible Programming in Statistics
\end{center}
\end{frame}


%---- ToC ----%

\begin{frame}{Outline}

\tableofcontents[]
\end{frame}

% ----------- section equations ---------- %
\section{Working with equations}

\begin{frame}{Working with equations}
We define a set of equations as
\begin{equation}
a = b + c^2,    
\end{equation}
\begin{equation}
a - c^2 = b,    
\end{equation}
\begin{equation}
\text{left side} = \text{right side},     
\end{equation}
\begin{equation}
\text{left side} + \text{something} \ge \text{right side},    
\end{equation}

for all something $>$ 0.
\end{frame}

\subsection{Aligning the same equations}

\begin{frame}{Aligning the same equations}
Aligning the equations by the equal sign gives a much better view into the placementsof the separate equation components.
\begin{align} 
a &= b + c^2, \\ 
a - c^2 &= b, \\
\text{left side} &= \text{right side},\\
\text{left side} + \text{something} &\ge \text{right side},    
\end{align}
\end{frame}

\subsection{Omit equation numbering}

\begin{frame}{Omit equation numbering}
Alternatively, the equation numbering can be omitted.
\begin{align*} 
a &= b + c^2, \\ 
a - c^2 &= b, \\
\text{left side} &= \text{right side},\\
\text{left side} + \text{something} &\ge \text{right side},    
\end{align*}

\end{frame}

\subsection{Ugly alignment}

\begin{frame}{Ugly alignment}
Ugly alignmentSome components do not look well, when aligned.  Especially equations with differentheights and spacing.  For example,
\begin{center}
 \begin{align} 
E &= mc^2, \\ 
m &= \frac{E}{c^2}, \\
c &= \sqrt{\frac{E}{m}}.
\end{align}
Take that into account.
\end{center}
\end{frame}


\section{Discussion}
\begin{frame}{Discussion}
This is where you’d normally give your audience a recap of your talk, where you coulddiscuss e.g.  the following
 \begin{itemize}
\item Your main findings
\item The consequences of your main findings
\item Things to do
\item Any other business not currently investigated, but related to your talk
\end{itemize}

\end{frame}


\end{document}